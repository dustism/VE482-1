\documentclass{article}
\usepackage{enumerate}
\usepackage{amsmath}
\usepackage{amssymb}
\usepackage{graphicx}
\usepackage{subfigure}
\usepackage{geometry}
\usepackage{caption}
\usepackage{indentfirst}
\usepackage[colorlinks,urlcolor=blue]{hyperref}
\usepackage{minted}
\usemintedstyle{autumn}
\setminted{linenos,breaklines,tabsize=4,xleftmargin=1.5em}
\geometry{left=3.0cm,right=3.0cm,top=3.0cm,bottom=4.0cm}
\title{VE482 Lab 3}
\author{Liu Yihao 515370910207}
\date{}

\begin{document}
\maketitle

\section{Simple git}
\begin{itemize}
\item Search what is \mintinline{c}{git} 

Git is a fast, scalable, distributed revision control system with
       an unusually rich command set that provides both high-level
       operations and full access to internals.
\item Install a git client.
\begin{minted}{shell}
apt-get install git
\end{minted}
\item Search the use of the following git commands:
\begin{itemize}
\item help Display help information about Git.
\item init Create an empty Git repository or reinitialize an existing one.
\item checkout Switch branches or restore working tree files.
\item branch List, create, or delete branches.
\item push Update remote refs along with associated objects.
\item pull Fetch from and integrate with another repository or a local branch.
\item merge Join two or more development histories together.
\item add Add file contents to the index.
\item diff Show changes between commits, commit and working tree, etc.
\item tag Create, list, delete or verify a tag object signed with GPG.
\item log Show commit logs.
\item fetch Download objects and refs from another repository.
\item commit Record changes to the repository.
\item clone Clone a repository into a new directory.
\item reset Reset current HEAD to the specified state.
\end{itemize}
\item Setup your git repository on the VE482 server.
\begin{minted}{shell}
git remote add ve482 git@ve482:515370910207/p1
git push ve482
\end{minted}
\end{itemize}

\section{Git game}

\section{Working with source code}

\subsection{The \mintinline{c}{rsync} command}
\begin{itemize}
\item In Minix 3 install the \mintinline{c}{rsync} software
\begin{minted}{shell}
pkgin install rsync
\end{minted}
\item Install \mintinline{c}{rsync} on you Linux system
\begin{minted}{shell}
apt-get install rsync
\end{minted}
\item Read \mintinline{c}{rsync} manpage
\begin{minted}{shell}
man rsync
\end{minted}
\item Create an exact copy of the directory \mintinline{c}{/usr/src} into the directory \mintinline{c}{/usr/src_orig}
\begin{minted}{shell}
cp -r /usr/src /usr/src_orig
\end{minted}
\item If you have altered Minix 3 source code during homework 2 remove your changes from \mintinline{c}{/usr/src_orig}
\item Create an exact copy of the Minix 3 directory \mintinline{c}{/usr/src_orig} into your Linux system, using \mintinline{c}{rsync} and \mintinline{c}{ssh} (note that the ssh server must be activated under Linux)
\begin{minted}{shell}
rsync -avz minix:/usr/src_orig/ ~/minix
\end{minted}
\end{itemize}

\subsection{The \mintinline{c}{diff} and \mintinline{c}{patch} commands}
\begin{itemize}
\item Read the manpages of \mintinline{c}{diff} and \mintinline{c}{patch}
\begin{minted}{shell}
man diff
man patch
\end{minted}
\item Using the \mintinline{c}{diff} command, create a patch corresponding to your changes in homework 2

\end{itemize}

\end{document}
